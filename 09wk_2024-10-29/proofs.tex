\documentclass{article}
\usepackage{amsmath}
\usepackage{amssymb}

\begin{document}

\title{Proof of Maximum Graph Value for a Divided Cycle}
\author{Warvin Hassan}
\date{}
\maketitle

\section*{Proof}

Let \( G = C_n + e \) be a graph formed by adding an extra edge \( e \) to an \( n \)-cycle \( C_n \). This additional edge \( e \) connects two non-adjacent vertices in \( C_n \), thereby creating a "divided cycle." Assign weights \( w_1, w_2, \ldots, w_n \) to the edges of \( C_n \) and a weight \( w_{e} \) to the additional edge \( e \). Our goal is to show that the maximum graph value \( V(G) \), which is the sum of the products of the edge weights across all spanning trees of \( G \), is achieved when the smallest weight is assigned to \( e \), with the remaining weights distributed among the edges of \( C_n \).

The graph value \( V(G) \) is defined as
\[
V(G) = \sum_{T \in \mathcal{T}(G)} \prod_{e \in T} w(e),
\]
where \( \mathcal{T}(G) \) denotes the set of all spanning trees in \( G \) and \( w(e) \) represents the weight of edge \( e \).

To calculate \( V(G) \), we consider two types of spanning trees in \( G \): those that include the extra edge \( e \) and those that exclude it. If a spanning tree includes \( e \), it will consist of \( e \) and \( n - 1 \) additional edges chosen from \( C_n \) to span all vertices. If a spanning tree excludes \( e \), it must be formed entirely from the edges in \( C_n \), which results in a spanning tree that is equivalent to a spanning tree of \( C_n \) itself.

We first calculate the number of spanning trees in each case. For the cycle \( C_n \), there are exactly \( n \) spanning trees, as each spanning tree of a cycle graph is formed by removing one of the \( n \) edges in \( C_n \). In \( G \), we have two types of spanning trees: those that include \( e \), of which there are \( n - 1 \), and those that exclude \( e \), of which there are \( n \). Therefore, \( G \) has a total of \( 2n - 1 \) spanning trees, with \( n \) spanning trees that exclude \( e \) and \( n - 1 \) spanning trees that include \( e \). Consequently, spanning trees that exclude \( e \) form the majority of spanning trees in \( G \).

Next, we analyze the effect of weight assignment on the graph value \( V(G) \). Consider the configuration where the smallest weight \( w_{\min} \) is assigned to \( e \), while the remaining, larger weights \( w_1, w_2, \ldots, w_n \) are assigned to the edges of \( C_n \). We now evaluate the contributions to \( V(G) \) from the two cases of spanning trees under this weight assignment.

For spanning trees that include \( e \), each tree’s contribution to the graph value will contain the term \( w_{\min} \) due to the inclusion of \( e \), resulting in a product of the form \( w_{\min} \prod_{i=1}^{n-2} w(e_i) \), where the product is taken over the \( n - 2 \) edges chosen from the \( n-1 \) edges of \( C_n \). Therefore, the total contribution from spanning trees that include \( e \) is
\[
\sum_{T \in \mathcal{T}_1} w_{\min} \prod_{e \in T \setminus \{e\}} w(e),
\]
where \( \mathcal{T}_1 \) is the set of spanning trees in \( G \) that include \( e \). Since \( w_{\min} \) is small, this contribution is minimized.

For spanning trees that exclude \( e \), each tree consists solely of edges from \( C_n \), and hence each tree’s product is of the form \( \prod_{e \in T} w(e) \), where \( T \) is a spanning tree of \( C_n \). The total contribution from these spanning trees is
\[
\sum_{T \in \mathcal{T}_2} \prod_{e \in T} w(e),
\]
where \( \mathcal{T}_2 \) denotes the set of spanning trees in \( C_n \). This contribution is maximized by placing the larger weights on the edges of \( C_n \), as each product \( \prod_{e \in T} w(e) \) will be maximized when the weights are higher.

Since the majority of the spanning trees in \( G \) are those that exclude \( e \) (namely, \( n \) out of \( 2n - 1 \)), their contribution has a larger impact on the total graph value \( V(G) \) than the contribution from the fewer spanning trees that include \( e \). By assigning the smallest weight to \( e \), we minimize the terms where \( e \) is present and allow the majority of spanning trees (those of \( C_n \)) to contribute maximally to the graph value by utilizing the larger weights.

Therefore, the maximum graph value \( V(G) \) is achieved by assigning the smallest weight to the extra edge \( e \) while placing the remaining weights on the edges of the cycle \( C_n \).

\(\square\)

\end{document}
